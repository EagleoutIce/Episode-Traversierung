\errorcontextlines 999999
\usepackage[ngerman]{babel}

\usepackage[%
    sopra-listings={encoding,cpalette,highlights},%
    sopra-tables, color-palettes={addons},%
    lecture-bibliography={biber,style=numeric-comp},%
    util, lithie-boxes={germanenv,koma,overwrite},%
    lithie-task-boxes={cpalette}, lecture-links={patchurl},%
    lecture-personal-resize,
    lecture-registers={disable}% would interfere with beamer
]{lithie-util}

\usepackage{forest}
\usetikzlibrary{overlay-beamer-styles}

\makeatletter
\sol@list@define@styles{%
  {keywordA: \@declaredcolor{sol@colors@lst@keywordA}\bfseries},%
  {keywordC: \@declaredcolor{sol@colors@lst@keywordB}\bfseries},%
}
\makeatother

\RestyleAlgo{plain}
\lstset{lineskip=5.5pt}
\lstfs{10}

\DefinePalette{Traversals}
{Orange,orangenfarben: RGB(255, 131, 53)}% Cadmium orange
{Grün,grünlich: RGB(123, 181, 46)}
{Gelb,gelblich: RGB(255, 200, 87)}
{Blau,bläulich: RGB(0, 114, 187)}
\SetShadeContrast{45}
\UsePalette{Traversals}

\usetheme[libs,nofootfade,centerfoot]{dividing-lines}
\SetColorProfile*{paletteA}{paletteB}{paletteC}

\usetikzlibrary{arrows.meta,decorations,decorations.pathreplacing,shapes.multipart,tikzmark,shapes.symbols}

\def\info#1{\bgroup\scriptsize\textcolor{gray}{(#1)}\egroup}
\SetAllLinkStyle{#1}
\def\fillfont{\def\mdseries@sf{medium}\sffamily}
\colorlet{lgray}{lightgray!48!white}
\tikzset{
    ldesc/.style={gray,font=\sffamily\sbfamily},
    lrel/.style={fill=white,rounded corners,minimum width=28mm,minimum height=7.5mm,align=center},
    lrel2/.style={fill=white,rounded corners,minimum width=28mm,minimum height=7.5mm*2,align=center},
    lsf/.style={fill=white,rounded corners,minimum width=28mm,minimum height=7.5mm*2,align=center,
        rectangle split, rectangle split parts=2},
    blob/.style={circle,draw, minimum size=1.9em,align=center},
    lblob/.style={blob,writ,font=\fillfont,#1},%
    lblob/.default={fill=lgray, draw=lgray,text=black},
    every picture/.append style={line join=round,line cap=round},%
    every node/.append style={font=\sffamily},%
    lblock/.style={block,writ,font=\fillfont,#1},%
    lblock/.default={fill=lgray, draw=lgray,text=black},
    block/.style={rectangle,draw, align=center, minimum height=1.25em,rounded corners=1.55pt},%
    K/.style n args={3}{alt=<#3>{shadeGray,opacity=.5}{},alt=<#2>{shadeA}{},alt=<#1>{shadeB}{}},
    K*/.style n args={2}{alt=<#2>{shadeGray,opacity=.5}{},alt=<#1>{shadeA}{}},
    T/.style n args={3}{alt=<#1>{fill=shadeB,draw=shadeB}{},alt=<#2>{fill=shadeA,draw=shadeA,font=\sbfamily}{},alt=<#3>{fill=shadeGray,draw=shadeGray,opacity=.5}{}},
}

\newcommand\parallelcontent[3][t]{%
    \begin{columns}[#1]
    \begin{column}{.475\linewidth}#2\end{column}\hfill
    \begin{column}{.475\linewidth}#3\end{column}
    \end{columns}
}

\usepackage[glows]{tikzpingus}
\usetikzlibrary{decorations.text,graphs}
\hypersetup{colorlinks=false}

\title{Traversierungsvarianten}
\subtitle{Verwandte Besuchen: Wie, wann und wo ein gutes Kind zu\\den geliebten Eltern rennt.}
\institute{SP, Universität Ulm}

\author{Florian Sihler}
\email{florian.sihler@uni-ulm.de}

\date{\today}
\outro{Ulm, \today}
\license[]{All rights reserved}

\def\PreTitlepage{\begingroup%
\let\oldinserttitle\inserttitle% allow it to be white on second slide
\let\oldinsertsubtitle\insertsubtitle% allow it to be white on second slide
\colorlet{PINGU@WHITE}{pingu@white}% hacksies for the whites
\only<2|handout:2>{\def\inserttitle{\color{pingu@white}\oldinserttitle}\def\insertsubtitle{\textcolor{pingu@white}{\oldinsertsubtitle}}}%
\onslide<2|handout:2>{%
\savebox0{\tikz{\pingu[lightsaber left=paletteC!90!white,left item angle=-20,right eye wink,monocle left,right wing grab,bow tie=paletteA!90!black, pants=paletteA,body=paletteA!20!black,lightsaber left length=2.35cm,lightsaber left outer glow factor=.09,hat]}}%
\begin{tikzpicture}[overlay,remember picture]%
    % beamer does not support changes of full background easily. so we do hacksies
    \pgfonlayer{background}
    \path[fill,black!99] (current page.north west) rectangle (current page.south east);
    \endpgfonlayer
    \node[above right=.15cm,scale=.65,xshift=.15cm] (pingu) at(current page.south west) {\usebox0};
    \node[below right,pingu@white,text width=.6\paperwidth,align=flush left] at(pingu.north east){Officially supported by the Ping-to-the-u-\allowbreak Foundation for Emotional Support. Look out for others so they may waddle with you!};
\end{tikzpicture}}%
}
\def\PostTitlepage{\endgroup}

\addbibresource{./references.bib}


\setcounter{tocdepth}{4}
\newsavebox\pinguA \newsavebox\pinguB \newsavebox\pinguC \newsavebox\pinguD
\newcommand*\mb[1]{$\underset{\text{\color{shadeD}\T{\strut\Large\textbullet}}}{\text{#1}}$}

\usepackage[link]{qrcode}
\outroright{\smash{\raisebox{1.33cm/2}{\qrcode[height=1.33cm]{https://github.com/EagleoutIce/Episode-Traversierung}}}\begin{tikzpicture}[remember picture,overlay]
    \node[above left,btdl@color@white,scale=.475] at (current page.south east) {\href{https://github.com/EagleoutIce/Episode-Traversierung}{Slides and \LaTeX-sources on GitHub!}};
\end{tikzpicture}}

\def\treewidth{\linewidth}
\newcommand*\tree[2][]{\downsize{\treewidth}{\begin{forest}for tree={lblob,minimum width=2.5em,s sep=4em-level*.5em,edge={line width=3pt,lgray,line cap=butt},#1}#2\end{forest}}}
\forestset{w/.style={fill=shadeGray,draw=shadeGray,text opacity=.5,edge={shadeGray}},m/.style={fill=shadeA,draw=shadeA,edge={shadeGray}},g/.style={fill=shadeA,draw=shadeA}}%

\hfuzz=6cm